\documentclass[12pt,english]{article}

\usepackage{geometry}
\geometry{letterpaper, margin=0.75in}

\usepackage{setspace}
\onehalfspacing

% sections on new page
\usepackage{titlesec}
\newcommand{\sectionbreak}{\clearpage}

\pagestyle{headings}

\usepackage{graphicx}
\usepackage{url}
\usepackage{hyperref}
\hypersetup{
    bookmarksnumbered,
    colorlinks,
    linkcolor=[rgb]{.4,0,0.4}
}
\usepackage{amssymb,amsmath}
\usepackage{subfig}
\usepackage{fancyvrb}

\usepackage[T1]{fontenc} % properly copyable underscore characters
\usepackage[usenames]{xcolor} % colored text options
\usepackage{color}

% code listings
\usepackage{listings}
\lstset{ %
  %basicstyle=\tiny,
  language=python,
  commentstyle=\itshape\color{blue},
  keywordstyle=\color{purple},
  identifierstyle=\color{orange},
  stringstyle=\color{red},
  keepspaces=true,
  showstringspaces=false,
  columns=flexible,
  % avoid turning code minus into hyphen em dash:
  literate=*{-}{{-}}1
}

%%%%%%%%%%%%%%%%%%%%
% boxed environments
\usepackage{framed}

% for configuration files
\newenvironment{cfgnote}[1][{}]{
\lstset{language=python}
\begin{oframed}
{\color{gray}\# configuration example: #1}

}{\end{oframed}}

% for code internals
\newenvironment{codenote}[1][{}]{
\lstset{language=python}
\begin{oframed}
{\color{blue}// code note: #1}

}{\end{oframed}}

% detail warnings
\newenvironment{warning}[1][{}]{
\definecolor{shadecolor}{rgb}{1,1,.7}
\begin{shaded}
{\color{red}! warning: #1}

}{\end{shaded}}

% hints
\newenvironment{hint}[1][{}]{
%\colorlet{shadecolor}{lightgray}
\definecolor{shadecolor}{rgb}{.9,.9,.9}
\begin{shaded}
{\color{purple}--- Hint: #1}

}{\end{shaded}}

%%%%%%%%%%%%%%%%
% To-do comment, with second optional "asignee" argument
% e.g. \todo{somebody ought to do this} or \todo[MPM]{Michael ought to do this}
\newcommand{\todo}[2][ToDo]{{\color{purple} [\textbf{#1}: #2]}}

%%%%%%%%%%%%%%%%
\newcommand{\cd}[1]{\texorpdfstring{{\color{blue} \texttt{#1}}}{#1}}
\newcommand{\cmd}[1]{`\cd{#1}'}
\newcommand{\eg}{{\em e.g.}}
\newcommand{\ie}{{\em i.e.}}
\newcommand{\tld}{\textasciitilde}
\newcommand{\vc}[1]{\ensuremath{\vec{\mathbf{#1}}}}
\newcommand{\floor}[1]{\ensuremath{\left\lfloor #1 \right\rfloor}}
\newcommand{\avg}[1]{\ensuremath{\left\langle #1 \right\rangle}}

\begin{document}

\title{\cd{JobManager.py} batch job management}
\author{Michael P. Mendenhall (LLNL)}
\maketitle

\cd{JobManager.py} provides helper utilities for submitting, tracking,
    and managing jobs through various batch submission system ``back ends,''
    or even managing parallel jobs on your local machine.
A local \cd{sqlite3} database file holds descriptions of jobs to run,
    along with tracking information on job completion status.

\tableofcontents

%%%%%%%%%%%%%%%%%%%%%%%%%%%%%%
%%%%%%%%%%%%%%%%%%%%%%%%%%%%%%
%%%%%%%%%%%%%%%%%%%%%%%%%%%%%%
\section{File locations}

\begin{hint}[include JobManager on your paths]
To let \cd{python3} find the necessary include files,
    the directory containing the job manager scripts should be added to
    the \cd{\$PYTHONPATH} environment variable, by: \newline
\cd{export PYTHONPATH=\$PYTHONPATH:<path to>/JobManager/} \newline
which you can place in your login script (\eg\ \cd{\tld/.bashrc}) to be set
    whenever you open a new terminal.
Similarly, to let you run \cd{JobManager.py} from any directory without
    giving its full path, you can add its directory to your executables lookup \cd{\$PATH}: \newline
\cd{export PATH=\$PATH:<path to>/JobManager/}
\end{hint}

\cd{JobManager} relies on an \cd{sqlite3} database file for its stored information,
    along with a directory for batch job submission scripts and logfiles.
By default, \cd{JobManager} will create a base directory at \cd{\$HOME/\$\{HOSTNAME\}\_jobs/},
    containing database file \cd{jdb.sqlite3}.

\begin{hint}[\cd{\$JOBSOUT} directory]
Configure an environment variable you can use to reference the JobManager output direcory by \newline
\cd{export JOBSOUT=\$HOME/\$\{HOSTNAME\}\_jobs/}
\end{hint}

%%%%%%%%%%%%%%%%%%%%%%%%%%%%%%
%%%%%%%%%%%%%%%%%%%%%%%%%%%%%%
%%%%%%%%%%%%%%%%%%%%%%%%%%%%%%
\section{Adding jobs to DB}

\begin{hint}[\cd{JobManager.py} arguments summary]
For a brief summary of \cd{JobManager.py} command-line options, run: \newline
\cd{JobManager.py {-}{-}help}
\end{hint}

Jobs can be added to the DB (but not submitted to run, yet) by running
    \cd{JobManager.py} with some combination of the following commands:

Contents of a job (or multiple jobs from a file) are specified by come combination of:
\begin{itemize}
    \item \cd{{-}{-}jobfile <input file>}: submit one job to run each line in file
    \item \cd{{-}{-}script}: single job script will be supplied on \cd{stdin} (typically useful for other scripts)
    \item \cd{{-}{-}name <job name>}: optional user-assigned name for job(s)
    \item \cd{{-}{-}walltime <time, s>}: maximum wall run time per job, in seconds (leave some excess --- batch systems may kill jobs exceeding this) \cd{{-}{-}cores <number>}: number of cores needed by each job; defaults to 1
    \item \cd{{-}{-}test <number>}: add this many test jobs.
        Each job is \cd{echo "Hello world <n>!"; sleep 5; echo "Goodbye!"};
        every fifth job is set to return a \cd{99} failure exit code.
    \item \cd{{-}{-}queue <qname>}: submission queue name, \eg\ \cd{pbatch},
        if system default is inappropriate (or missing).
        Set to \cd{local} to force running on the local machine, instead of through batch system.
    \item \cd{{-}{-}account <acct>}: cluster resource bank allocation account.
        If unspecified, user defaults may apply (OK if you only have one bank allocation on system).
    \item \cd{{-}{-}db <db file>}: optional non-default jobs DB location
\end{itemize}

%%%%%%%%%%%%%%%%%%%%%%%%%%%%%%
%%%%%%%%%%%%%%%%%%%%%%%%%%%%%%
%%%%%%%%%%%%%%%%%%%%%%%%%%%%%%
\section{Launching jobs}

After adding jobs to the DB (or in the same command as job submission),
    interaction with the batch queue system is controlled by:
\begin{itemize}
    \item \cd{{-}{-}limit <number>}: set limit to this many active submissions to batch system.
        If unspecified, the initial default is the number of processor cores available on the local machine
            --- much less than available on a batch system.
        When specified, the \cd{cores} resource availability is saved to the DB, and persists
        in future runs until changed (or the DB is re-created).
    \item \cd{{-}{-}status}: check and display status of previously-submitted jobs
    \item \cd{{-}{-}launch}: check status of previously-submitted jobs, and submit new jobs (up to \cd{limit})
    \item \cd{{-}{-}cycle <wait time, s>}: repeat \cd{launch} operation until all jobs are completed,
        with specified wait time in seconds between repeats
    \item \cd{{-}{-}trickle <time, s>}: used with \cd{launch} or \cd{cycle}, spaces out job launch start times
        by specified number of seconds for a softer ``trickle launch'' of processes
        (potentially useful if jobs briefly compete for access to a slow resource at the same time)
\end{itemize}

\begin{hint}[Local test run]
To try running a batch of test jobs on the local machine, generate them with: \newline
\cd{JobManager.py {-}{-}test 20 {-}{-}queue local} \newline
then launch them, and wait for completion with status checks every 5 seconds: \newline
\cd{JobManager.py {-}{-}launch {-}{-}cycle 5} \newline
You could also combine the arguments from both commands above into a single \cd{JobManager.py} call.
The process should have created a \cd{\$JOBSOUT/test/} directory,
    containing the scripts, logfile, and output status of each test job.
You can also view the Jobs DB contents by \newline
\cd{sqlite3 \$JOBSOUT/jdb.sqlite3 'select job\_id, status, t\_submit, use\_walltime, return\_code from jobs;'}
\end{hint}

\begin{hint}[Raise batch job submission limit]
For submitting jobs to a cluster batch queue system, you should raise the limit on how many jobs
    \cd{JobManager.py} is willing to submit for running at one time by, \eg\newline
\cd{JobManager.py {-}{-}limit 2000} \newline
and let the custer batch management system decide on its own how many of these
    to actually let run (depending on your account's priority)
\end{hint}

%%%%%%%%%%%%%%%%%%%%%%%%%%%%%%
\subsection{Bundling jobs}

\cd{JobManager.py} provides options for ``bundling'' multiple jobs together into a single batch queue submission.
This is useful if you have lots of very short jobs:
    the batch system will probably take much longer to allocate resources and run 12000 one-minute jobs instead of
    400 30-minute jobs (each performing 30 one-minute tasks in a row).
This feature can also be used to parallelize multiple single-core jobs on batch systems
    allocating multi-core nodes.
\cd{JobManager.py} still tracks the individual completion status of each job within the
    single ``bundled'' job submitted to the queue system.

Job bundling needs to be done somewhere between adding the individual jobs to the Jobs DB
    and launching them on the batch DB.
The bundling commands can be run as a separate step between submission and launch,
    or combined with the command-line arguments for those tasks.
To bundle jobs, give \cd{JobManager.py} the argument \cd{{-}{-}bundle <total time in seconds>}.
This will scan all not-yet-launched runs in the database, and pack as many of them as possible
    into groups up to the total bundle limit.

\begin{hint}[Bundled test run]
To test bundling with a single command: \newline
\cd{JobManager.py {-}{-}test 100 {-}{-}bundle 600 {-}{-}queue local {-}{-}launch {-}{-}cycle 5} \newline
which will bundle the 100 initial nominally-60-second jobs into groups of 10$\cdot$\cd{ncores}
    (where \cd{ncores} depends on the local machine), to be run as (nominally) 10-minute multi-core jobs.
\end{hint}

%%%%%%%%%%%%%%%%%%%%%%%%%%%%%%
%%%%%%%%%%%%%%%%%%%%%%%%%%%%%%
%%%%%%%%%%%%%%%%%%%%%%%%%%%%%%
\section{Monitoring progress}

Along with the \cd{status} and \cd{cycle} commands above,
    the output of each running job is directed to a logfile specified in the DB.
Default logfiles are placed in \cd{\$JOBSOUT/<jobname>/log\_<jobnum>.txt}.

%%%%%%%%%%%%%%%%%%%%%%%%%%%%%%
%%%%%%%%%%%%%%%%%%%%%%%%%%%%%%
%%%%%%%%%%%%%%%%%%%%%%%%%%%%%%
\section{Jobs database}

Job status is stored in an \cd{sqlite3} database file, with table structure defined by \cd{JobsDB\_Schema.sql}.
The table \cd{jobs} describes the jobs being managed; table \cd{resources} names resource categories jobs may require,
    and \cd{resource\_use} associates individual jobs with resource requirements.



%%%%%%%%%%%%%
%%%% FIN %%%%
%%%%%%%%%%%%%
\end{document}


